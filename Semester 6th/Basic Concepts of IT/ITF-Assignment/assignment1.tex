\pagebreak
\section{Question 1 \normalsize(10 pts)} 

\begin{question}
	You know that information and technology are distinct concepts that combine to form the concept of information technology.
	\begin{enumerate}
		\item What are the definitions and distinctions of information technology, and how do they fit together to form IT?
		\item Provide an example that demonstrates this distinction and supports your previous answer conceptually.
		\item Mention some of the ways that information technology has had a significant impact on the modern market and how it's contributed to the success of organizations worldwide.
	\end{enumerate}
%You know that information and technology are distinct concepts that combine to form the concept of information technology. Now, could you provide an example that demonstrates the distinction between information and technology and how they combine to form information technology as a hot concept?
\end{question}

\pagebreak
\begin{answer}

\end{answer}

\pagebreak
\section{Question 2 \normalsize(50 pts)}

\begin{question}
	Assume you are starting your new business:
	\begin{enumerate}
		\item Provide a detailed and creative description of your company, including its name, the product or service it provides, its target market, the types of employees required, and the resources needed.
		\item Identify the customers, possible suppliers, and stakeholders involved in the business.
		\item Analyze the potential impacts of implementing a CRM, SCM, and ERP system in this business in detail.
		\item Explain how these systems can improve efficiency, productivity, and customer satisfaction based on the example provided.
		\item Create a graph or chart to illustrate the relationships between these systems in the analysis (like what was provided in the lecture notes).
	\end{enumerate}
%	Give a detailed description of your company, including its name, the product or service you offer, your target market, the types of employees needed, and the resources required. Then mention your customers, suppliers, and the people who are stakeholders in your business. Next, analyze the potential impacts of implementing a CRM, SCM, and ERP system in your business, including their interrelationships, and how they could improve efficiency, productivity, and customer satisfaction based on the example you have provided. Use a graph or chart to visually depict the relationships between these systems in your analysis.

	Make sure to cover all the elements of your business that were questioned while also being creative. Each entity, correlation, and your impeccable description of each, as well as the quality and correctness of your graph, will be considered, so give it your best shot!
\end{question}

\pagebreak
\begin{answer}
	
\end{answer}

\pagebreak
\section{Question 3 \normalsize(25 pts)}

\begin{question}
	Walmart uses several information systems to manage its supply chain, including an advanced logistics system that tracks the movement of goods from suppliers to stores and an inventory management system that monitors product availability. By ensuring that products are always in stock and available for purchase, they have been able to achieve high levels of efficiency, reduce costs, and improve customer satisfaction through the use of information systems.
	
	It's your turn now!
	\begin{enumerate}
		\item Describe a specific company that has successfully implemented information systems in its supply chain. (You may use the company from your previous question to guide your description.)
		\item Explain how information systems increase customer satisfaction, decrease costs, and improve decision-making processes through supply chain management.
		\item Provide a technical and conceptual overview of the benefits of information systems in the supply chain management of this company.
	\end{enumerate}
%	Describe a specific company in detail and imagine that it has implemented information systems successfully in its supply chain; you may use the company from your previous question to guide your description. Then, explain meticulously, both technically and conceptually, how information systems can increase customer satisfaction, decrease costs, and improve decision-making processes in this company through supply chain management.
	
\end{question}

\pagebreak
\begin{answer}
	
\end{answer}

\pagebreak
\section{Question 4 \normalsize(15 pts)}

\begin{question}
	On slide number 36, you learned about six different aspects of an ERP system; now:
	\begin{enumerate}
		\item Provide at least six distinct examples of real-world companies, platforms, or applications in Iran or outside Iran that have functionalities similar to the six aspects. (Attach the URL of your chosen companies' websites to their names in your text.)
		\item How do these companies, platforms, and applications use these functionalities to support their business operations?
	\end{enumerate}
%	Now, can you provide examples of real-world companies, platforms, or applications in Iran that have functionalities similar to these six aspects and explain how they use them to support their business operations?
\end{question}

\pagebreak
\begin{answer}
	
\end{answer}

\pagebreak
\section{Question 5 \normalsize(20 pts)}

\begin{question}
	The Iris dataset contains information about three types of Iris flowers and includes the following characteristics:
	\begin{itemize}
		\item Sepal length
		\item Sepal width
		\item Petal length
		\item Petal width
	\end{itemize}
	
	Load the Iris dataset from the sklearn library first. Now calculate the following measures for column 1 of this dataset:
	\begin{enumerate}
		\item Mean
		\item Median
		\item Mode
		\item The first quartile (Q1)
		\item The third quartile (Q3)
		\item Normalize the data in this column using the min-max normalization method.
	\end{enumerate}
\end{question}
\pagebreak
\begin{answer}
	Codes should be addressed, answers should be explained, and a conclusion should be given.
	\begin{enumerate}     
		\item ... 
		
		\item ... 
	\end{enumerate}
\end{answer}

\pagebreak
\section{Question 6 \normalsize(50 pts)}

\begin{question}
	In this exercise, you will use Python to work with a dataset to perform data cleaning and preprocessing tasks, such as replacing missing values with the mean and exploring alternative strategies for dealing with missing data.
	\begin{enumerate}
		\item Download the dataset that is provided with the exercise.
		\item Open the dataset using the pandas library.
		\item Obtain a summary of the statistical measures of this dataset using the describe() function.
		\item In this dataset, a value of zero represents missing data. In this step, display the number of zero values in columns 1 to 5.
		\item Replace the zero values with NaN in columns 1 to 5.
		\item Repeat step 4 using the isnull() function.
		\item Show the first ten rows of the dataset.
		\item Create a copy of columns 1 to 5 of the original dataset. Remove records with missing values. Display the shape of the dataset before and after this change.
		\item Create another copy of columns 1 to 5 of the original dataset. Replace missing values with the mean value of the dataset. Display the number of missing values in each column using the isnull() function.
		\item Create another copy of columns 1 to 5 of the original dataset. Replace missing values with the mean value of each column using the SimpleImputer function from the sklearn library. Finally, display the number of missing values in each column using the isnull() function.
		\item What are other strategies for replacing missing values? Display the output of each method.
	\end{enumerate}
\end{question}

\pagebreak
\begin{answer}
	Each step must include a screenshot, a code snippet and its explanation, or a data file.
	\begin{enumerate}     
		\item ... 
		
		\item ... 
	\end{enumerate}
\end{answer}

\pagebreak
\section{Question 7 \normalsize(30 pts)}

\begin{question}
	Boxplots are useful tools for identifying outliers in a dataset. In this context, we will look at how to use Python boxplots to detect outliers in the Iris dataset.
	\begin{enumerate}
		\item Load the Iris dataset using Python and create a dataframe from it. The columns of the dataframe, from left to right, are named as follows:
		\begin{itemize}
			\item sepal\_length
			\item sepal\_width
			\item petal\_length
			\item petal\_width
		\end{itemize}
		\item Generate a boxplot for the petal\_length field of the Iris dataset using the matplotlib library.
		\item Execute the describe() function on the values of the petal\_length field and display the output.
		\item Remove any outliers from the petal\_length field.
	\end{enumerate}
\end{question}

\pagebreak
\begin{answer}
	Each step must include a screenshot, a code snippet and its explanation, or a data file.
	\begin{enumerate}     
		\item ... 
		
		\item ... 
	\end{enumerate}
\end{answer}
%
%\pagebreak
%\section{Bonus Question \normalsize(20 pts)}
%
%\begin{question}
%	In this question, you must create a brief video clip introducing an IT concept!
%	
%	On the second slide of the first chapter, you were introduced to some examples of how information technology is heavily integrated into our daily lives, electronically, digitally, and virtually. Now, choose one of the free topics among them and submit it with your name on \href{https://docs.google.com/spreadsheets/d/1Goh9vyUes10Tv58EJAerfda9urhe5pgntkUJA4cgsgw/edit?usp=sharing}{this Google Sheets link.} Then, explain the concept in detail and show us an example of its usage in our daily lives that is not really known but is significant to us.
%	
%	You may deliver your presentation on your preferred platform, but your grade is determined by the quality of your presentation as well as the richness of your material. Please keep your video between 8 and 10 minutes in length, and then share the video links with your friends on the same Google Sheets where you registered your topic, so everyone can watch it and become familiar with that concept!
%\end{question}
